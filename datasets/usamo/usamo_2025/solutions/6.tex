Define the $P$ person's partitions as bubbles. Then we get a matching between the $n$ people $P$ and the first person's $n$ bubbles $B$. Connect a person to a bubble if they value that bubble as more than $1$, so one person is connected to all the bubbles on this bipartite graph $G \subset P \times B$. If Hall's condition is satisfied on these people, take that matching to win. Else, there exists some set $X$ of people such that they match less than $|X|$ bubbles. Remove that set, and repeatedly remove any set with $|N_G(X)| < |X|$. We must end up with some non-empty set of people $M$ such that $M$ and $N_G(M)$ have a matching, and no person in $P \setminus M$ matches any bubble in $N_G(M)$. In this case, we may take a matching on $M$ and $N_G(M)$, and remove this matching and the corresponding bubbles. Since the remaining people aren't connected to these bubbles, this inductively preserves the score condition for them. This is further explained below:


Call the $s$ people who you are about to delete settled. For the people who aren't settled, you need to make sure they lose exactly $s$ groups in the deletion, and that each of their remaining groups still has value at least 1. For each of the $s$ settled people, delete the group they match with one at a time. Call the group we are about to delete $G$. By the hypothesis (this is what the Hall's was for), the $n-s$ unsettled people value $G$ at most $1$. In particular, none of these unsettled people will possess a group that is contained strictly within $G$. They either own two adjacent groups whose border lies inside $G$, or they own one group that contains $G$ (or is equal to $G$).
Suppose an unsettled person falls under the former case. Before, the two adjacent groups had a total value of at least $2$, and after deleting $G$, at most $1$ total value of cupcake was deleted from the union of those groups. Gluing the remains of those two groups together thus creates a new group with total value at least $1$, maintaining the condition. In particular, gluing these groups together decreases this unsettled person's total number of groups by one, which is what we want.
Suppose an unsettled person falls under the latter case. So, by deleting $G$, we put a hole in this unsettled person's group. Then, they simply glue the two remains of this group together with another group to the left. The value of this new group is at least $1$ (in fact, equality holds only when there the remains of the group totaled $0$ in value).
We obviously don't need to worry about regrouping the settled people. So, repeating this process $s$ times on each of the settled people will leave us with only the $n-s$ unsettled people remaining, each of whom have $n-s$ groups.

This allows you to reformulate the problem with $n-s$ people, and the solution follows by induction.