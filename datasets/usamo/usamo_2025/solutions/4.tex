Consider the line perpendicular to $BC$ at $C$. Let this line meet $(ABC)$ at $C'$. Let $M$ be the midpoint of $CC'$. We claim $M$ is the circumcenter of $AFP$. Note that $P$ lies on $(ABC)$.

Clearly $AC'CP$ is an isosceles trapezoid, so the perpendicular bisector of $AP$ meets $CC'$ at $M$. Now, since $\angle BCC' = 90$, $\angle C'AB = 180 - \angle BCC' = 90$, so $AC' \mid\mid CF$. Let the reflection of $H$ over $F$ be $H'$, and notice $H'$ lies on $(ABC)$. Then the perpendicular bisector of $AF$ is just the midline of $AC'CH'$, so it also passes through $M$. Therefore, $M$ is the circumcenter of $AFP$.

Now, clearly $\triangle MCX \cong \triangle MCY$, so $CX = CY$ and we are done.